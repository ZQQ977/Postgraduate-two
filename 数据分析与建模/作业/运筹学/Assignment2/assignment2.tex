\documentclass[12pt]{article}
\usepackage[UTF8]{ctex}
\usepackage{mathtools}
\usepackage{graphicx}
\usepackage[margin=1in]{geometry} 
\usepackage{amsmath,amsthm,amssymb}
\renewcommand{\vec}[1]{\boldsymbol{#1}}
\newcommand{\N}{\mathbb{N}}
\newcommand{\Z}{\mathbb{Z}}
 
\newenvironment{theorem}[2][Theorem]{\begin{trivlist}
\item[\hskip \labelsep {\bfseries #1}\hskip \labelsep {\bfseries #2.}]}{\end{trivlist}}
\newenvironment{lemma}[2][Lemma]{\begin{trivlist}
\item[\hskip \labelsep {\bfseries #1}\hskip \labelsep {\bfseries #2.}]}{\end{trivlist}}
\newenvironment{exercise}[2][Exercise]{\begin{trivlist}
\item[\hskip \labelsep {\bfseries #1}\hskip \labelsep {\bfseries #2.}]}{\end{trivlist}}
\newenvironment{problem}[2][Problem]{\begin{trivlist}
\item[\hskip \labelsep {\bfseries #1}\hskip \labelsep {\bfseries #2.}]}{\end{trivlist}}
\newenvironment{question}[2][Question]{\begin{trivlist}
\item[\hskip \labelsep {\bfseries #1}\hskip \labelsep {\bfseries #2.}]}{\end{trivlist}}
\newenvironment{corollary}[2][Corollary]{\begin{trivlist}
\item[\hskip \labelsep {\bfseries #1}\hskip \labelsep {\bfseries #2.}]}{\end{trivlist}}

\newenvironment{solution}{\begin{proof}[Solution]}{\end{proof}}

\begin{document}

\title{Assignment 2}
\author{183139-耿冬冬}
\date{\today}
\maketitle

\section{}
\noindent
(1)反证法:\\
假设$f(x)$在$c \in (a,b)$取得最大值,则$f(c)>f(a)$且$f(c)>f(b)$。
则存在一点$\xi \in(a,c),f^{'}(\xi)>0$,存在一点$\eta \in(c,b),f^{'}(\eta)<0$,则$f^{'}(\eta) < f^{'}(\xi)$\\
由凸函数的性质可知:一元可微凸函数,一阶导数单调不减,则有$f^{'}(\eta) \geq f^{'}(\xi)$,所以假设不成立,$f(x)$的最大值只能在边界取得。\\
\begin{figure}
    \centering
    \includegraphics[scale=0.4]{凸函数.png}
\end{figure}
(2)反证法:\\
假设$f(x)$在$c \in (a,b)$取得最大值,则$f(c)>f(a)$且$f(c)>f(b)$。\\
则存在$t \in(0,1)$,使得$c=ta+(1-t)b$。\\根据凸函数的定义可知$f(ta+(1-t)b)\leq tf(a)+(1-t)f(b)$。\\假设$f(a)\leq f(b)$,则$tf(a)+(1-t)f(b)\leq tf(b)+(1-t)f(b)=f(b)$。\\
所以$f(c)=f(ta+(1-t)b)\leq f(b)$,假设不成立,命题得证。
\section{}
\noindent
由题意可知:\\
\[\frac{\partial f(x)}{\partial x_1}=3x_1^2-x_2-2=0\] 
\[\frac{\partial f(x)}{\partial x_2}=-x_1+2x_2+3=0\] 
解得:$x_1=\frac{1}{2},x_2=-\frac{5}{4}$,\quad $x_1=-\frac{1}{3},x_2=-\frac{5}{3}$。
\[\frac{\partial f^2(x)}{\partial x_1^2}=6x_1 \quad \frac{\partial f^2(x)}{\partial x_1x_2}=-1\]
\[\frac{\partial f^2(x)}{\partial x_2x_1}=-1 \quad \frac{\partial f^2(x)}{\partial x^2_2}=2\]
所以黑塞矩阵为:\\
\[\boldsymbol{H}=\begin{bmatrix}
    6x_1 & -1 \\
    -1 & 2 
\end{bmatrix}
\]
当$x_1=\frac{1}{2},x_2=-\frac{5}{4}$时,黑塞矩阵正定,所以取极小值。\\
当$x_1=-\frac{1}{3},x_2=-\frac{5}{3}$时,黑塞矩阵负定,所以取极大值。
\section{}
\noindent
\[\frac{\partial f(x)}{\partial x_1}=2x_2x_3-4x_3+2x_1-2\]
\[\frac{\partial f(x)}{\partial x_2}=2x_1x_3-2x_3+2x_2-4\]
\[\frac{\partial f(x)}{\partial x_3}=2x_1x_2-4x_1-2x_2+2x_3+4\]
于是有:
\[\frac{\partial f^2(x)}{\partial x_1^2}=2 \quad \frac{\partial f^2(x)}{\partial x_1x_2}=2x_3 \quad \frac{\partial f^2(x)}{\partial x_1x_3}=2x_2-4\]
\[\frac{\partial f^2(x)}{\partial x_2x_1}=2x_3 \quad \frac{\partial f^2(x)}{\partial x_2^2}=2 \quad \frac{\partial f^2(x)}{\partial x_1x_3}=2x_1-2\]
\[\frac{\partial f^2(x)}{\partial x_3x_1}=2x_2-4 \quad \frac{\partial f^2(x)}{\partial x_3x_2}=2x_1-2 \quad \frac{\partial f^2(x)}{\partial x_3^2}=2\]
黑塞矩阵如下:
\[\boldsymbol{H}=\begin{bmatrix}
    2 & 2x_3 & 2x_2-4 \\
    2x_3 & 2 & 2x_1-2 \\
    2x_2-4 & 2x_1-2 & 2
\end{bmatrix}\]
将各点带入判断黑塞矩阵的正定性得:\\
点$(0,3,1)$的黑塞矩阵为负定,取极大值。点$(0,1,-1)$的黑塞矩阵为负定,取极大值。\\
点$(1,2,0)$的黑塞矩阵为正定,取极小值。点$(2,1,1)$的黑塞矩阵为负定,取极大值。\\
点$(2,3,-1)$的黑塞矩阵为负定,取极大值。
\section{}
\noindent
\[f(x)=-x_1^2-2x_2^2+2x_1x_2+2x_2\]
\[\frac{\partial f(x)}{\partial x_1}=-2x_1+2x_2\]
\[\frac{\partial f(x)}{\partial x_2}=-4x_2+2x_1+2\]
\begin{equation*}
    \begin{split}
        g(\alpha)&=f(x-\alpha \triangledown f(x))\\
        &=-[x_1-\alpha (-2x_1+2x_2)]^2-2[x_2-\alpha (-4x_2+2x_1+2)]^2+\\
          &\quad 2[x_1-\alpha (-2x_1+2x_2)][x_2-\alpha (-4x_2+2x_1+2)]+2[x_2-\alpha (-4x_2+2x_1+2)]
    \end{split}
\end{equation*}
得:
\begin{equation*}
    \begin{split}
        g^{'}(\alpha)=&2(-2x_1+2x_2)[x_1-\alpha (-2x_1+2x_2)]+4(-4x_2+2x_1+2)[x_2-\alpha (-4x_2+2x_1+2)]\\
        &-2(-2x_1+2x_2)[x_2-\alpha (-4x_2+2x_1+2)]-2(-4x_2+2x_1+2)[x_1-\alpha (-2x_1+2x_2)]\\
        &-2(-4x_2+2x_1+2)
    \end{split}
\end{equation*}
第一次迭代:\\
$\frac{\partial f(x)}{\partial x_1}=0 \quad \frac{\partial f(x)}{\partial x_2}=2 \quad g(\alpha)=-8\alpha ^2-4\alpha 
\quad g^{'}(\alpha)=-16\alpha-4=0 \quad \alpha=-\frac{1}{4}$\\
得:$x_1=0 \quad x_2=\frac{1}{2}$\\
第二次迭代:\\
$\frac{\partial f(x)}{\partial x_1}=1 \quad \frac{\partial f(x)}{\partial x_2}=0 \quad g(\alpha)=-\alpha ^2-\frac{1}{2}-\alpha+1 
\quad g^{'}(\alpha)=-2\alpha-1=0 \quad \alpha=-\frac{1}{2}$\\
得:$x_1=\frac{1}{2} \quad x_2=\frac{1}{2}$\\
第三次迭代:\\
$\frac{\partial f(x)}{\partial x_1}=0 \quad \frac{\partial f(x)}{\partial x_2}=1 \quad g(\alpha)=-\frac{1}{4}-2(\frac{1}{2}-\alpha)^2+(\frac{1}{2}-\alpha)+2(\frac{1}{2}-\alpha)
\\ g^{'}(\alpha)=4(\frac{1}{2}-\alpha)-3=0 \quad \alpha=-\frac{1}{4}$ \\
得:$x_1=\frac{1}{2} \quad x_2=\frac{3}{4}$ \\
第四次迭代:\\
$\frac{\partial f(x)}{\partial x_1}=\frac{1}{2} \quad \frac{\partial f(x)}{\partial x_2}=0
\quad g(\alpha)=-(\frac{1}{2}-\frac{1}{2}\alpha)^2-2(\frac{3}{4})^2+\frac{3}{2}(\frac{1}{2}-\frac{1}{2}\alpha)+\frac{3}{2}
\\ g^{'}(\alpha)=\frac{1}{2}(1-\alpha)-\frac{3}{4}=0 \quad \alpha=-\frac{1}{2}$\\
得:$x_1=\frac{3}{4} \quad x_2=\frac{3}{4}$
\section{}
\noindent
根据题意由拉格朗日乘子法得:
\[L(x,\lambda)=4(x_1-2)^2+3(x_2-4)^2+\lambda _1(x_1+x_2-5)+\lambda _2(1-x_1)+\lambda _3(2-x_2)\]
得:
\[   \frac{\partial L}{\partial x_1}=8(x_1-2)+\lambda _1-\lambda _2 \] 
\[   \frac{\partial L}{\partial x_2}=6(x_2-4)+\lambda _1-\lambda _2 \] 
\[   \frac{\partial L}{\partial \lambda _1}=x_1+x_2-5 \] 
\[   \frac{\partial L}{\partial \lambda _2}=1-x_1 \] 
\[   \frac{\partial L}{\partial \lambda _3}=2-x_2 \] 
由KKT条件可得:\\
(1)\[\frac{\partial L}{\partial x_1}=8(x_1-2)+\lambda _1-\lambda _2=0\]
\[   \frac{\partial L}{\partial x_2}=6(x_2-4)+\lambda _1-\lambda _3=0\] 
(2) 
\[ \lambda _1,\lambda _2,\lambda _3 \geq 0\]
(3)
\[\lambda _1(x_1+x_2-5)=0  \quad \lambda _2(1-x_1)=0 \quad \lambda _3(2-x_2)=0 \]
(4)
\[x_1+x_2-5 \leq 0 \quad 1-x_1\leq 0 \quad 2-x_2\leq 0\]
若$\lambda _1=0 \quad \lambda _2=0 \quad \lambda _3=0$,则条件(4)不成立。\\
若$\lambda _1 \neq 0 \quad \lambda _2=0 \quad \lambda _3=0$,则$x_1=\frac{11}{7} \quad x_2=\frac{24}{7}$。\\
若$\lambda _1=0 \quad \lambda _2 \neq 0 \quad \lambda _3=0$,则条件(2)不成立。\\
若$\lambda _1\neq 0 \quad \lambda _2\neq 0 \quad \lambda _3=0$,则条件(1)不成立。\\
若$\lambda _1=0 \quad \lambda _2=0 \quad \lambda _3 \neq 0$,则条件(2)不成立。\\
若$\lambda _1 \neq 0 \quad \lambda _2=0 \quad \lambda _3\neq 0$,则条件(2)不成立。\\
若$\lambda _1=0 \quad \lambda _2 \neq 0 \quad \lambda _3\neq 0$,则条件(2)不成立。\\
若$\lambda _1\neq 0 \quad \lambda _2\neq 0 \quad \lambda _3\neq 0$,则条件(3)不成立。\\
综上所述,$f(x)$的最小值在$\boldsymbol{X}=(\frac{11}{7},\frac{24}{7})^T$处取得,$\min f(x)=\frac{12}{7}$。
\section{}
\section{}
\noindent
(1)
\[\frac{\partial z(x)}{\partial x_1}=2x_1-3 \quad \frac{\partial z(x)}{\partial x_2}=2x_2-4\]
第一次迭代:$\triangledown z(x^1)=(-\frac{5}{2},-\frac{7}{2})$
\[ \min g(y)=-\frac{5}{2}y_1-\frac{7}{2}y_2 \]
\begin{equation*}s.t.
\left\{
    \begin{array}{l}
        y_1+y_2 \leq 1 \\
        y_1,y_2 \geq 0
    \end{array}
\right.
\end{equation*}
得$y^1=(0,1)$。$x^1+\alpha (y^1-x^1)=(\frac{1}{4},\frac{1}{4})+\alpha [(0,1)-(\frac{1}{4},\frac{1}{4})]
=(\frac{1}{4}-\alpha \frac{1}{4},\frac{1}{4}+\alpha \frac{3}{4})$\\
$\min f(x^1-\alpha (y^1-x^1))=(\frac{1}{4}-\alpha \frac{1}{4})^2+(\frac{1}{4}+\alpha \frac{3}{4})^2-3(\frac{1}{4}-\alpha \frac{1}{4})-4(\frac{1}{4}+\alpha \frac{3}{4}) \quad \alpha \in [0,1]$\\
得$\alpha =1$,$x^2=(0,1)$。\\
第二次迭代:$\triangledown z(x^2)=(-3,-2)$\\
\[ \min g(y)=-3y_1-2y_2 \]
\begin{equation*}s.t.
\left\{
    \begin{array}{l}
        y_1+y_2 \leq 1 \\
        y_1,y_2 \geq 0
    \end{array}
\right.
\end{equation*}
得$y^2=(1,0)$。$x_2+\alpha (y^2-x^2)=(0,1)+\alpha [(1,0)-(0,1)]=(\alpha ,1-\alpha)$\\
$\min f(x_2+\alpha (y^2-x^2))=\alpha ^2+(1-\alpha)^2-3\alpha -4(1-\alpha)\quad \alpha \in (0,1)$
得$\alpha=\frac{1}{4}$,$x^3=(\frac{1}{4},\frac{3}{4})$\\
第三次迭代:$\triangledown z(x^3)=(-\frac{5}{2},-\frac{5}{2})$ \\
\[ \min g(y)=-\frac{5}{2}y_1-\frac{5}{2}y_2 \]
\begin{equation*}s.t.
\left\{
    \begin{array}{l}
        y_1+y_2 \leq 1 \\
        y_1,y_2 \geq 0
    \end{array}
\right.
\end{equation*}
已经迭代至最优解。\\
(2)
\[L(x,\lambda)=x_1^2+x_2^2-3x_1-4x_2+\lambda _1(x_1+x_2-1)+\lambda _2(-x_1)+\lambda _3(-x_2)\]
得:
\[\frac{\partial L}{\partial x_1}=2x_1-3+\lambda _1-\lambda _2\]
\[\frac{\partial L}{\partial x_2}=2x_2-4+\lambda _1-\lambda _3\]
<1>
\[2x_1-3+\lambda _1-\lambda _2=0 \quad 2x_2-4+\lambda _1-\lambda _3=0\]
<2>
\[\lambda _1,\lambda _2,\lambda _3 \geq 0\]
<3>
\[\lambda _1(x_1+x_2-1)=0 \quad \lambda _2(-x_1)=0 \quad \lambda _3(-x_2)=0\]
<4>
\[x_1+x_2-1 \leq 0 \quad -x_1 \leq 0 \quad -x_2 \leq 0\]
带入点$(\frac{1}{4},\frac{3}{4})$,得$\lambda _1=\frac{5}{2} \quad \lambda _2=0 \quad \lambda _3=0$,满足KKT的所有条件。
\section{}
\noindent
根据题意可得:
\begin{equation*}
\left\{
    \begin{array}{l}
        2+x_1^{2}=1+3x_2 \\
        x_1+x_2=x_3=q 
    \end{array}
\right. 
\end{equation*}\\
解得:$x_1=\frac{-3+\sqrt{12q+5}}{2} \quad x_2=q-\frac{-3+\sqrt{12q+5}}{2} \quad x_3=q$,\\
$\quad t(x_1)=t(x_2)=\frac{6q+11-3\sqrt{12q+5}}{2} \quad t(x_3)=3+q$。
\section{}
\noindent
(1)由图可知A到D的有三条不同的路线,\\
即$A\to B\to D \quad A\to C\to D \quad A\to C\to B\to D$。\\
$\hat{t_1}(x_1)=t(x_1)+x_1t^{'}(x_1)=21+0.02x_1 \quad \hat{t_2}(x_2)=8+0.2x_2 \\
 \hat{t_3}(x_3)=4+0.04x_3 \quad \hat{t_4}(x_4)=19+0.02x_4 \quad \hat{t_5}(x_5)=6+0.2x_5$
\begin{equation*}
    \left\{
        \begin{array}{l}
            21+0.02f_1+6+0.2f_1=8+0.2f_2+19+0.02f_2=8+0.2f_3+4+0.04f_3+6+0.2f_3\\
            f_1+f_2+f_3=4
        \end{array}
        \right.
\end{equation*}
$f_1=f_2=0 \quad f_3=4$\\
(2)去掉link C—B后变为两条路线。\\
\begin{equation*}
    \left\{
        \begin{array}{l}
            21+0.02f_1+6+0.2f_1=8+0.2f_2+19+0.02f_2 \\
            f_1+f_2=4
        \end{array}
        \right.
\end{equation*}
$f_1=f_2=2$\\
$27.44*2*2>(18+0.44*4)*4$,出行总成本增加。


\end{document}

%g^{'}(\alpha) = 2(-2x_1+2x_2)[x_1-\alpha (-2x_1+2x_2)]+4(-4x_2+2x_1+2)[x_2-\alpha (-4x_2+2x_1+2)]-2(-2x_1+2x_2)[x_2-\alpha (-4x_2+2x_1+2)]-2(4x_2+2x_1+2)[x_1-\alpha (-2x_1+2x_2)]