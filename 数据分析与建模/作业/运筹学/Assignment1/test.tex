% --------------------------------------------------------------
% This is all preamble stuff that you don't have to worry about.
% Head down to where it says "Start here"
% --------------------------------------------------------------
 
\documentclass[12pt]{article}

\usepackage[UTF8]{ctex}
\usepackage{mathtools}
\usepackage{amsmath} % used for boldsymbol.
\renewcommand{\vec}[1]{\boldsymbol{#1}} % Uncomment for BOLD vectors.

\usepackage[margin=1in]{geometry} 
\usepackage{amsmath,amsthm,amssymb}
 
\newcommand{\N}{\mathbb{N}}
\newcommand{\Z}{\mathbb{Z}}
 
\newenvironment{theorem}[2][Theorem]{\begin{trivlist}
\item[\hskip \labelsep {\bfseries #1}\hskip \labelsep {\bfseries #2.}]}{\end{trivlist}}
\newenvironment{lemma}[2][Lemma]{\begin{trivlist}
\item[\hskip \labelsep {\bfseries #1}\hskip \labelsep {\bfseries #2.}]}{\end{trivlist}}
\newenvironment{exercise}[2][Exercise]{\begin{trivlist}
\item[\hskip \labelsep {\bfseries #1}\hskip \labelsep {\bfseries #2.}]}{\end{trivlist}}
\newenvironment{problem}[2][Problem]{\begin{trivlist}
\item[\hskip \labelsep {\bfseries #1}\hskip \labelsep {\bfseries #2.}]}{\end{trivlist}}
\newenvironment{question}[2][Question]{\begin{trivlist}
\item[\hskip \labelsep {\bfseries #1}\hskip \labelsep {\bfseries #2.}]}{\end{trivlist}}
\newenvironment{corollary}[2][Corollary]{\begin{trivlist}
\item[\hskip \labelsep {\bfseries #1}\hskip \labelsep {\bfseries #2.}]}{\end{trivlist}}

\newenvironment{solution}{\begin{proof}[Solution]}{\end{proof}}
 
\begin{document}
 
% --------------------------------------------------------------
%                         Start here
% --------------------------------------------------------------
 
\title{Assignment 1}
\author{183139-耿冬冬}

\maketitle
\section{}
\quad \\
设Television Day Time Unit为$x_1$,Prime Time Unit为$x_2$,Radio Unit为$x_3$,Magazines Unit为$x_4$.

\[ \max z=400000 x_{1}+900000 x_{2}+500000 x_{3}+200000 x_{4} \]
\[
\left\{\begin{array}{l}
    {4000 x_{1}+75000 x_{2}+3000 x_{3}+15000 x_{4} \leqslant 800000} 
    \\ {300000 x_{1}+400000 x_{2}+200000 x_{3}+100000 x_{4} \geqslant 200000} 
    \\ {4000 x_{1}+75000 x_{2} \leq 500000}
    \\ {x_{1} \geqslant 3 \quad x_{2} \geqslant 2
    \\ {5 \leqslant x_{3},x_{4} \leqslant 10}   
\end{array}\right.
\]
\section{}
\quad \\
产销平衡的运输问题。\\
设$w_{i}$到$s_{j}$的运输量为$x_{i j}$.设$a_{i}$为$w_i$的容量,$n_j$为$s_j$的需求量,$c_{i j}$为$w_{i}$到$s_{j}$的cost.其中$i=1,2,3,\quad j=1,\dots5$.
根据题意可知
\[ a=[100,200,50]  \quad n=[80,90,70,60,50]\]
\[
c=\left[ \begin{array}{ccccc}{1} & {2} & {4} & {3} & {6} \\ {5} & {2} & {4} & {4} &{4}\\ {1} & {1} & {1} & {3} & {2}\end{array}\right]
\]

\[ \begin{array}{l}{\min z=\sum_{i=1}^{3} \sum_{j=1}^{5} c_{i j} x_{i j}} \] 

\[ 
\left\{\begin{array}{ll}{\sum_{j=1}^{5} x_{i j} = a_{i}} & {(i=1,2,3)} \\ {\sum_{i=1}^{3} x_{i j}=n_{j}} & {(j=1,2, \cdots, 5)} \\ {x_{i j} \geqslant 0} & {(i=1,2,3 ; j=1,2, \cdots, 5)}\end{array}\right.
\]

\section{}
\quad \\
(1)$\max z=x \quad  s.t.\{ 1 \leqslant x \leqslant 2 \quad x < 0\}$, remove constraint 2 ,it is feasible.\\
(2)$\max z=x \quad s.t.\{ x \geqslant 2 \quad x \leqslant 1\}$, remove constraint 2, it is unbounded.\\
(3)同(1)\\
(4)$\max z=x+y \quad s.t.\{x+y \leqslant 1 \quad x+y \geqslant 2 \quad x,y \geqslant 0\}$, remove constraint 2, it has an infinite number of optimal solutions.\\
(5)No Exist.\\
(6)$\max z=x \quad  s.t.\{ 1 \leqslant x \leqslant 2 \}$, add constraint $\{x<0\}$, it is infeasible.\\
(7)$\max z=x+y \quad s.t.\{2x+y \leqslant 2 \quad x,y \geqslant 0\}$, add constraint $\{x+y \leqslant 1\}$, it has an infinite number of optimal solutions.\\
(8)$\max z=x+y \quad s.t.\{x+y \leqslant 1 \quad x,y \geqslant 0\}$, add constraint $\{x+2y \leqslant 1\}$, it has exactly one optimal solution.\\
(9)No Exist.\\
(10)No Exist.\\
(11)$\min z=x \quad s.t.\{x \geqslant 1\}$, change objective function $\max z=x$, it is unbounded.\\
(12)$\max z=x \quad s.t.\{x \geqslant 1\}$, change objective function $\min z=x$, it has exactly one optimal solution.\\
(13)$\max z=x+y \quad s.t.\{x+y \geqslant 1 \quad x,y \geqslant 0\}$, change objective function $min z=x+y$, it has an infinite number of optimal solutions.\\

\section{}
\quad \\
标准型为:
\[\max z=-x_1-x_2+4x_3+0x_4+0x_5+0x_6\]
\[
\left\{\begin{array}{l}
    x_1+x_2+2x_3+x_4=9\\
    x_1+x_2-x_3+x_5=2\\
    -x_1+x_2+x_3+x_6=4\\
    x_1,x_2 \cdots x_6 \geqslant 0   
\end{array}\right.
\]
\\
simplex table如下:
{
\begin{table}[h]
    \centering
    \begin{tabular}{lll|llllll|l}
    \multicolumn{3}{l|}{}                              & -1 & -1 & 4  & 0 & 0 & 0  &  \\ \hline
    \multicolumn{1}{l|}{$C_B$} & \multicolumn{1}{l|}{x_B} & b & $x_1$ & $x_2$ & $x_3$ & $x_4$ & $x_5$ & $x_6$ &  \\ \hline
    \multicolumn{1}{l|}{0} & \multicolumn{1}{l|}{$x_4$} & 9 & 1  & 1  & 2  & 1 & 0 & 0  & $\frac{9}{2}$ \\
    \multicolumn{1}{l|}{0} & \multicolumn{1}{l|}{$x_5$} & 2 & 1  & 1  & -1 & 0 & 1 & 0  & - \\
    \multicolumn{1}{l|}{0} & \multicolumn{1}{l|}{$x_6$} & 4 & -1 & 1  & [1]  & 0 & 0 & 1  & 4 \\ \hline
    \multicolumn{3}{l|}{$c_j-z_j$}                              & -1 & -1 & 4  & 0 & 0 & 0  &  \\ \hline
    \multicolumn{1}{l|}{0} & \multicolumn{1}{l|}{$x_4$} & 1 & [3]  & -1 & 0  & 1 & 0 & -2 & $\frac{1}{3}$ \\
    \multicolumn{1}{l|}{0} & \multicolumn{1}{l|}{$x_5$} & 6 & 0  & 2  & 0  & 0 & 1 & 1  & - \\
    \multicolumn{1}{l|}{4} & \multicolumn{1}{l|}{$x_3$} & 4 & -1 & 1  & 1  & 0 & 0 & 1  & - \\ \hline
    \multicolumn{3}{l|}{$c_j-z_j$}                              & 3  & -5 & 0  & 0 & 0 & -4 &  \\ \hline
    \multicolumn{1}{l|}{-1} & \multicolumn{1}{l|}{$x_1$} & $\frac{1}{3}$ & 1   &  $-\frac{1}{3}$  &  0  &  $\frac{1}{3}$ & 0  &  $-\frac{2}{3}$  &  \\
    \multicolumn{1}{l|}{0} &  \multicolumn{1}{l|}{$x_5$} & 6 &  0  &   2 &   0 &  0 &  1 &  1  &  \\
    \multicolumn{1}{l|}{4} &  \multicolumn{1}{l|}{$x_3$}& $\frac{13}{3}$  &  0  &  $\frac{2}{3}$  &  1  & $\frac{1}{3}$  &  0 &  $\frac{1}{3}$  &  \\ \hline
    \multicolumn{3}{l|}{$c_j-z_j$}                              &  0  &  -4  &   0 &  -1 &  0 &  -2  &  \\ \hline
    \end{tabular}
    \end{table}
}\\
所以当$X=(\frac{1}{3},0,\frac{13}{3},0,0,0)^{T}$时,$z$取最小值为$-17$.

\section{} 
\quad \\
(1)由题意可得(1): $x_1+x_3-x_4=3+3\beta$, (2):$x_2-x_3=1-\beta$. 由此可得simplex table with $X=(x_1,x_2)^{T}$如下:
{
\begin{table}[h]
    \centering
    \begin{tabular}{lll|llll|l}
    \multicolumn{3}{l|}{}                            & $\alpha$  & 2 & 1  & -4 &  \\ \hline
    \multicolumn{1}{l|}{$C_B$} & \multicolumn{1}{l|}{$x_B$} & b & $x_1$  & $x_2$  &  $x_3$  &  $x_4$  &  \\ \hline
    \multicolumn{1}{l|}{$\alpha$} & \multicolumn{1}{l|}{$x_1$} & $3+3\beta$ & 1 & 0 & 1  & -1 &  \\
    \multicolumn{1}{l|}{2} & \multicolumn{1}{l|}{$x_2$} &  $1-\beta$& 0 & 1 & -1 & 0  &  \\ \hline
    \multicolumn{3}{l|}{$c_j-z_j$}                            &  0 & 0  &  $3-\alpha$  & $\alpha-4$   & 
    \end{tabular}
    \end{table}
}\\
(2)
由
\left\{\begin{array}{l}
3-\alpha \leqslant 0\\
\alpha -4 \leqslant 0 \\
\end{array}
\right,可得$\alpha$的取值为$3 \leqslant \alpha \leqslant 4$.\\
(3)
由\left\{\begin{array}{l}
    3+3\beta \geqslant 0 \\ 1- \beta \geqslant 0 \\
\end{array}\right,可得$\beta$的取值为$-1 \leqslant \beta \leqslant 1$

\section{}
\quad \\
(1)根据题意得simplex table如下:
\begin{table}[h]
    \centering
    \begin{tabular}{l|l|llllll|l|}
     &   & $x_1$ & $x_2$ & $x_3$  &  $x_4$ &  $x_5$  & $x_6$  &  RHS   \\ \hline
     & 1 &  -6 & 0 & 0 & 0 & -1 &   & -14 \\ \hline
    $x_6$ & 0 & 3 & 0 &  $-\frac{14}{3}$ & 0 &  1  & 1 & 7    \\
    $x_2$ & 0 & 6 & 1 &  2 & 0 &  $\frac{5}{2}$  & 0 & 5   \\
    $x_4$ & 0 & 0 & 0 &  $\frac{1}{3}$ & 1 &   0 & 0 & 0   \\ \hline
    \end{tabular}
    \end{table}
\\
得$a=7,b=-6,c=0,d=1,e=0,f=\frac{1}{3},g=0$.\\
(2)由simplex table可知\\
\[
    \boldsymbol{B^{-1}} = \begin{bmatrix}
        3 & 0 & -\frac{14}{3} \\
        6 & 1 & 2 \\
        0 & 0 & \frac{1}{3}
    \end{bmatrix}
\]\\
(3)$\quad \frac{\partial x_2}{\partial x_1}=-6 \quad \frac{\partial z}{\partial x_5}=-1 \quad \frac{\partial x_6}{\partial b_3}=0$ \\
(4)$\quad \boldsymbol a_5 = \boldsymbol a_6 + \frac{5}{2}\boldsymbol a_2$

\section{}
\quad \\
(1)对偶问题为:
\[\max 6y_1+3y_2\]
\[
\left\{
    \begin{array}{l}
        2 y_1+y_2 \leqslant 3 \\
        -y_1+y_2 \leqslant 4 \\
        y_1+2y_2 \leqslant 6 \\
        6y_1+y_2 \leqslant 7 \\
        -5y_1+2y_2 \leqslant 1\\
        y_1,y_2 \geqslant 0
    \end{array}
    \right
\]\\
(2)由$w=(1,1)$可知 dual problem constraint 1和4为紧约束,由对偶问题的性质可得:
\[
\left\{
    \begin{array}{l}
        2x_1+6x_4=6\\
        x_1+x_4=3
    \end{array}
    \right
\]\\
解得原问题最优解为$X^{\star}=(3,0,0,0,0)^{T},\min =9$.
\section{}
\quad \\
(1)False\\
(2)False\\
(3)True\\
原问题和对偶问题解的关系如下表:\\
\begin{table}[h]
    \centering
    \begin{tabular}{l|l}
    \hline
    原问题 & 对偶问题 \\ \hline
    有可行解,且有最优解 & 有可行解,且有最优解  \\ \hline
    有可行解,但无最优解 & 无可行解 \\ \hline
    无可行解 & 无可行解\\ \hline
    无可行解 &  有可行解,但无最优解\\ \hline
    \end{tabular}
    \end{table}\\
\section{}
\quad \\
(1)
\[
 \boldsymbol a_2^{'}=\boldsymbol B^{-1}}\boldsymbol a_2=\begin{bmatrix}
    1 & 0 \\
    1 & 1 \\
\end{bmatrix} \begin{bmatrix}
    2 \\ 5
\end{bmatrix} = \begin{bmatrix}
    2 \\ 7
\end{bmatrix}
\]
\[
 \sigma_2 = C_2 - \boldsymbol C_B \boldsymbol B^{-1} \boldsymbol a_2=-5 < 0
\]\\
最优解不变,单纯型表如下:
\begin{table}[h]
    \centering
    \begin{tabular}{l|l|lllll|l}
     & z & $x_1$ & $x_2$ & $x_3$ & $x_4$ & $x_5$ & RHS \\ \hline
    z & 1 & 0 & -5 & -1 & -2 & 0 & -12 \\ \hline
    $x_1$ & 0 & 1 & 2 & 1 & 1 & 0 & 6 \\
    $x_5$ & 0 & 0 & 7 & 1 & 1 & 1 & 10 \\ \hline
    \end{tabular}
    \end{table}

(2)原问题变为unbounded,simplex table如下:
\begin{table}[h]
    \centering
    \begin{tabular}{l|l|l|lllll|l}
    \multicolumn{3}{l|}{} & 2  & -1 & 1 & 0 & 0 &  \\ \hline
    $C_B$  &  $x_B$ &  b  &  $x_1$  & $x_2$ & $x_3$ & $x_4$ & $x_5$ &  \\ \hline
    1      &  $x_3$    & 6     & 0  & 1  & 1 & 1 & 0 & - \\
    0      &    $x_5$  & 4     & -1 & 2  & 0 & 0 & 1 & - \\ \hline
    \multicolumn{3}{l|}{$c_j-z_j$} & 2  & -2 & 0 & 0 & 0 & 
    \end{tabular}
    \end{table}\\

(3)simplex table如下:
\begin{table}[h]
    \centering
    \begin{tabular}{l|l|l|lllll|l}
    \multicolumn{3}{l|}{} & 2  & -1 & 1 & 0 & 0 &  \\ \hline
    $C_B$  &  $x_B$ &  b  &  $x_1$  & $x_2$ & $x_3$ & $x_4$ & $x_5$ &  \\ \hline
    1      &  $x_3$    & 6     & 3  & 1  & 1 & 1 & 0 &  \\
    0      &    $x_5$  & 2     & 3 & 1  & 0 & 0 & 1 &  \\ \hline
    \multicolumn{3}{l|}{$c_j-z_j$} & -1  & -2 & 0 & -1 & 0 & 
    \end{tabular}
    \end{table}\\
最优解为$\boldsymbol X^{\star}=(0,0,6,0,2)^{T},\min =-6$\\
\\
\\
\\
\\
\\
\\
\\
\\
\\
\\
(4)simplex table如下:
\begin{table}[h]
    \centering
    \begin{tabular}{l|l|l|llllll|l}
    \hline
    \multicolumn{3}{l|}{} & 2 & -1 &  1 & 0 & 0 & -1 &   \\ \hline
    $C_B$   &   $x_B$ &  b  &  $x_1$ & $x_2$  &  $x_3$   & $x_4$ & $x_5$ & $x_6$ &   \\ \hline
    0      &   $x_4$  & 6   & [1]  & 1 & 1  & 1 & 0 & -1 & 6 \\
    0      &   $x_5$  & 4   & -1 & 2 & 0  & 0 & 1 & 2  & - \\ \hline
    \multicolumn{3}{l|}{$c_j-z_j$} & -2 & 1 & -1 & 0 & 0 & -1 &   \\ \hline
    2      &   $x_1$   & 6  &  1  & 1 & 1  & 1 &  0 & -1 & -  \\
    0      &   $x_5$   & 10  &  0  & 3 & 1  & 1 & 1 & [1]  & 10 \\ \hline
    \multicolumn{3}{l|}{$c_j-z_j$} & 0  & -3 &  -1 &  -2 & 0 &  1 &   \\ \hline
    2     &   $x_1$   & 16  &  1  & 4 & 2  & 2 &  1 & 0 &   \\
    -1      &   $x_6$   & 10  &  0  & 3 & 1  & 1 & 1 & 1  &  \\ \hline
    \multicolumn{3}{l|}{$c_j-z_j$} & 0  & -9 &  -3 &  -3 & -1 &  0 &   \\ \hline
    \end{tabular}
    \end{table}\\
最优解为$\boldsymbol X^{\star}=(16,0,0,0,0,10)^{T},\min =-22$.\\


% --------------------------------------------------------------
%     You don't have to mess with anything below this line.
% --------------------------------------------------------------
 
\end{document}
